\documentclass{epsconf}
\usepackage{graphicx}
\usepackage{wrapfig}
\usepackage{amsmath}
\usepackage[style=phys,articletitle=false,defernumbers=true,maxnames=2,firstinits=true,uniquename=init,backend=bibtex8,arxiv=abs,mcite]{biblatex}
\bibliography{biblio}
\renewcommand{\bibfont}{\normalfont\footnotesize}
\renewbibmacro{in:}{%
  \ifentrytype{article}{}{%
  \printtext{\bibstring{in}\intitlepunct}}}
\renewcommand\refname{\vskip -1cm}
\DefineBibliographyStrings{english}{%
  references = {},
}
%% to set line space in bibliography
\usepackage{setspace}
\usepackage{multicol}
\title{SOL transport and filamentary dynamics in tokamaks: from L to
  inter-ELM filaments in high density regimes}
\author{N. Vianello$^{1}$}
\institute{$^{1}$ Consorzio RFX (CNR, ENEA, INFN, Universit{\'a} di Padova, Acciaierie Venete SpA), C.so Stati Uniti 4, 35127, Padova, Italy}
\begin{document}
\maketitle
Addressing the role of Scrape Off Layer filamentary transport is
presently a subject of intense studies in fusion science. 
Intermittent coherent structures are found to dominate transport in
the L-Mode scenarios as but also to strongly contribute to particle
and energy losses in the ELM and inter-ELM phases in H-mode.
The role of convective radial losses towards the first
wall has become even more important due to their 
contribution 
to the so called process of \emph{shoulder formation} in
L-Mode, with the progressive flattening of the density
scrape off layer profile at high density
\cite{LaBombard:2001ks,Carralero:2015gu,Militello:2016hk,Vianello:2017ku}.
Investigation on this process revealed the strong
relationship between divertor conditions and the upstream profiles,
mediated by filaments dynamics which varies
accordingly to the modification of the downstream condition.
Preliminary investigation suggested that similar mechanisms are likely to
occur in H-Mode as well \cite{Carralero:2017gb} and even contribute to
the so-called H-mode density limit (HDL) \cite{bernert2014h}.  
The present contribution will report the results obtained in two
different tokamak, ASDEX-Upgrade and TCV,  coordinated on European
level, to investigate the role of filamentary transport in high-density
regimes both in L and H-mode. The combination 
of results from different machines enlarge the operational
space explored moving from a device with closed divertor,  metallic
first wall and an highly efficient cryogenic pumping system to a carbon
machine with a complete open divertor. The mechanism of shoulder
formation has been tested against variation of parallel connection
length through current scan at constant magnetic field in both the
devices, and by poloidal flux expansion variation in TCV. 

\begingroup
\setstretch{0.9}
\printbibliography
\endgroup

\end{document}
