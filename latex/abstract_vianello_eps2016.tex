\documentclass{epsconf}
\usepackage{graphicx}
\usepackage{wrapfig}
\usepackage{amsmath}
\usepackage[style=phys,articletitle=false,defernumbers=true,maxnames=2,firstinits=true,uniquename=init,backend=bibtex8,arxiv=abs,mcite]{biblatex}
\bibliography{biblio}
\renewcommand{\bibfont}{\normalfont\footnotesize}
\renewbibmacro{in:}{%
  \ifentrytype{article}{}{%
  \printtext{\bibstring{in}\intitlepunct}}}
\renewcommand\refname{\vskip -1cm}
\DefineBibliographyStrings{english}{%
  references = {},
}
%% to set line space in bibliography
\usepackage{setspace}
\usepackage{multicol}
\title{SOL transport and filamentary dynamics in tokamaks: from L to
  inter-ELM filaments in high density regimes}
\author{N. Vianello$^{1}$}
\institute{$^{1}$ Consorzio RFX (CNR, ENEA, INFN, Universit{\'a} di Padova, Acciaierie Venete SpA), C.so Stati Uniti 4, 35127, Padova, Italy}
\begin{document}
\maketitle
Addressing the role of Scrape Off Layer filamentary transport is
presently a subject of intense studies in fusion science. Indeed
intermittent coherent structures are found to dominate transport in
the L-Mode scenarios as well as in the inter-ELM phases.
The role of convective radial losses towards the first
wall has become even more important when it was recognized its
contribution 
to the so called process of \emph{shoulder formation} in
L-Mode, which describe the progressive flattening of the density
scrape off layer profile, experimentally observed at high density in a variety of
devices\cite{LaBombard:2001ks,Carralero:2015gu,Militello:2016hk,Vianello:2017ku}. Investigation on this process revealed the strong
inter-relationship existing between the divertor conditions, including
collisionality, neutral density, detachment degree, pumping efficiency and the upstream profiles,
relationship mediated by the dynamics of filaments which varies
accordingly to the modification of the downstream condition. Given the
relevancy of the topic a strong effort is on-going on European level
to coordinate the study of filaments and in particular of their
modification in high density regimes both in L and H-Mode. The present
contribution will provide a detailed investigation of the role of
filamentary transport in high density regimes both in L and in H-mode
as obtained in the ASDEX-Upgrade and TCV tokamaks. The combined
results from different machines allow to increase the operational
space explored moving from a device with closed divertor,  metallic
first wall and highly efficient cryogenic pumping system to a carbon
machine with a complete open divertor. We will provide evidences
concerning the relationship between shoulder formation and increase
filamentary transport in both L and H mode focusing on the role of
neutrals pressure, 

\begingroup
\setstretch{0.9}
\printbibliography
\endgroup

\end{document}
