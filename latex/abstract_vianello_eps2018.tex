\documentclass{epsconf}
\usepackage{graphicx}
\usepackage{wrapfig}
\usepackage{amsmath}
\usepackage[style=phys,articletitle=false,defernumbers=true,maxnames=2,firstinits=true,uniquename=init,backend=bibtex8,arxiv=abs,mcite]{biblatex}
\bibliography{biblio}
\renewcommand{\bibfont}{\normalfont\footnotesize}
\renewbibmacro{in:}{%
  \ifentrytype{article}{}{%
  \printtext{\bibstring{in}\intitlepunct}}}
\renewcommand\refname{\vskip -1cm}
\DefineBibliographyStrings{english}{%
  references = {},
}
%% to set line space in bibliography
\usepackage{setspace}
\usepackage{multicol}
\title{SOL transport and filamentary dynamics in high density tokamak regimes}
\author{N. Vianello}
\institute{Consorzio RFX, C.so Stati Uniti 4, 35127, Padova, Italy}
\begin{document}
\maketitle
Addressing the role of Scrape Off Layer filamentary transport is
presently a subject of intense studies in fusion science. 
Intermittent structures are found to dominate transport in
L-Mode and to strongly contribute to particle
and energy losses in H-mode as well.
The role of convective radial losses has become even more important due to its 
contribution 
to the  \emph{shoulder formation} in
L-Mode, describing the progressive flattening of the density
scrape off layer profile at high density
\cite{LaBombard:2001ks,Carralero:2015gu,Militello:2016hk,Vianello:2017ku}.
Investigation of this process revealed the strong
relationship between divertor conditions and the upstream profiles,
mediated by filaments dynamics which varies
according to the downstream conditions.
Preliminary investigations suggested that similar mechanisms
occur in H-Mode as well \cite{Carralero:2017gb} and contribute to
the so-called H-mode density limit (HDL) \cite{bernert2014h}.  
The present contribution will report the results obtained in a
coordinated effort between the ASDEX-Upgrade and TCV tokamaks, to address
the role of filamentary transport in high density regimes both in L
and H-Mode. The combined  results enlarges the operational
space explored, from a device with a closed divertor, metallic
first wall and cryogenic pumping system to a carbon
machine with a complete open divertor.
The mechanism of shoulder formation and the role of filamentary
transport have been tested against variation of parallel connection
length, through a current or flux expansion scan,
against magnetic topology, comparing single and double null plasmas
and against divertor neutral densities, through modification of
cryopump efficiency. Upstream profiles are found strongly resilient to
modification through flux expansion, or magnetic topology,
whereas I$_p$ scan at constant B$_t$ is found very efficient in
modifying  upstream profiles in both the devices. This is
accompanied by a variation of blob sizes which are larger at
smaller current for similar values of divertor collisionality in TCV.
On the other hand
fueling is insufficient to cause saturation of SOL
profiles in H-Mode since large neutral pressure is
needed. Consistently coherent fluctuations detected on J$_{sat}$ in
the inter-ELM phase in AUG are
found larger whenever the cryopumps is switched off.
This, together with observations from JET
\cite{wynn},
suggests that a proper way to control upstream profiles in
H-Mode can be based on pumping efficiency and design with respect to divertor
geometry. 
The resulting picture suggests a complex
relationship between divertor and upstream profiles, where filaments
are modified by divertor conditions as well as by neutral particles
interaction.

\begin{multicols}{2}%[\printbibheading]
\begingroup
\setstretch{0.7}
\printbibliography[heading=none]
\endgroup
\end{multicols}
\end{document}
